\documentclass[sigconf,authordraft]{acmart}

\begin{document}

\title{Guiding Synthesis by Generating Examples with Strings}

\author{Chunyu Xia}
\email{chunyuX@outlook.com}

\author{Zac Blanco}
\email{zblanco@ucsd.edu}

\begin{abstract}

A key challenge with program synthesis is generating enough examples in the set
of constraints such that the program conforms to a users desired specifications.
In this work we build a system described by \cite{laich2020guiding}. The
original work focused on synthesizing code for UI layouts for Android
applications. This work attempts to apply the concepts from that work into the
domain of strings on SyGuS problems.

\end{abstract}

\maketitle

\section{Introduction}

Program synthesis is designed to make the task of designing programs easier to
understand for humans. Humans are inherently bad at writing sound logic. One way
that synthesis aids is by making the decision about leaving the specific
instructions that need to be made by a program up to a computer. Instead, humans
can more clearly specify the desired behavior of a program by providing a set of
input and output examples which specify the program behavior.

For many cases providing input and output examples, works well. We can easily
specify the constraints on which a particular function must operate. The main
problem with this form of specification is that depending on the type of input;
strings, bitvectors, or integers the particular input space for a program can be
incredibly large. This leads to a large number of input specifications being
required in order to fully specify the desired behavior of a program.

SyGuS solvers such as Euphony \cite{lee2018accelerating} and EUsolver
\cite{alur2017scaling} both work by providing a set of input and output
examples. However, depending on the users' desired behavior for a number of
particular cases, it may be difficult to find a program where the user is
satisfied with the outputs across \textit{all} input cases.

The work done by Laich et al. in \cite{laich2020guiding} have shown that it is
possible to allow a user to guide synthesis by generating example programs,
and then using learned statistical methods to iteratively increase the size
of the specification in order to generate a more complete program which handles
many more input and output specifications than a user may be able to come up
with themselves.

In this work, we attempt to replicate the system devised by Laich et al, but
applied to the domain of strings using SyGuS grammars as used by EUSolver and
Euphony.

The paper is structured in the following manner: Section 2 talks more about the
motivation for this work. Section 3 discusses some of the related works in this
area of the field. Section 4 describes the system design and implementation.
Finally, section 5 evaluates our system.

\section{Motivation}


One of the main problems with generating programs using a set of constraints is
that the program is only guaranteed to work on the input and output constraints.
This is similar to the way neural networks are designed. By training or testing
with a particular set of data, the network (or synthesized program) is only
guaranteed to work for those inputs which it was trained on.

Synthesizing programs and deep learning methods with neural networks actually
share many of the same problems when it comes to training and performing well on
a limited set of data. As mentioned above, synthesizers using Programming By
Example (PBE) methods and neural networks are trained on a set of training data
but also need to generalize well across the entire input population which is
generally not available for training

Programs however also need to generalize to other examples. So, if there aren't
enough


\section{Related Works}

\section{System Design and Implementation}

\section{Evaluation}

\section{Conclusion}

\bibliographystyle{plain}
\bibliography{refs}

\end{document}
